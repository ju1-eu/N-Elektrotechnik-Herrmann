%ju 13-Aug-22 03-Hochvolt.tex
\textbf{Reichweite Batterie} $\sim 770~km$ (MERCEDES EQS (2021) -
S-KLASSE \footnote{\url{https://www.auto-motor-und-sport.de/})} (Stand:
Mai/2022)

\textbf{Spannung Bordnetz} $400~V - 800~V$ (Porsche Taycan, Audi
e-tron)

\textbf{Batterie laden} 1-Phase ($3,7~Km$), 3-Phasen, Gleichstrom

\section{Was sind HV-eigensichere
Fahrzeuge?}\label{was-sind-hv-eigensichere-fahrzeuge}

Gewährleisten einen vollständigen Berühr- und Lichtbogenschutz gegenüber
HV-System.

\begin{itemize}
\item
  System überwacht sich selbst
\item
  Serienfahrzeug
\item
  Sicherheitslinie (Pkw)

  \begin{itemize}
  \item
    \textbf{Außer:} Lkw, Busse, Eigenbau (Vorsicht beim Umrüsten und
    Nachrüsten), Unfallfahrzeug
  \end{itemize}
\end{itemize}

\textbf{IT-Netz} IT steht für isoliert und Terra (von der Erde
isoliertes System): Das HV-System ist sowohl wechsel- als auch
gleichstromseitig von der Fahrzeugmasse galvanisch getrennt oder
isoliert.

\section{Qualifikationsstufen}\label{qualifikationsstufen}

Jeder \textbf{Gewerbetreibende} benötigt die Stufe-S, um ein HV-Fahrzeug
bewegen zu dürfen.

\textbf{Unterweisung} einmal jährlich für Gesellen und halbjährlich für
Lehrlinge. $\to$ Person sensibilisieren auf jedes Fahrzeug einzeln.

\textbf{Wie heißt die neue Sicherheitsverordnung?}

DGUV 209-093 (Deutsche gesetzliche Unfallversicherung) Qualifizierung
für Arbeiten an Fahrzeugen mit HV-Systemen.

\textbf{Qualifikationsstufen}

\begin{itemize}
\item
  \textbf{Stufe S} sensibilisierte Person (Bedienen von Fahrzeugen mit
  HV-System)
\item
  \textbf{Stufe 1S} Fachkundige unterwiesene Person (Motto: Hände weg
  von Orange!)
\item
  \textbf{Stufe 2S} fachkundige Person (alle arbeiten, aber nicht unter
  Spannung, Freischalten)
\item
  \textbf{Stufe 3S} fachkundige Person -- für Arbeiten unter Spannung
  stehenden HV-System (Energiespeicher öffnen)
\end{itemize}

\textbf{Welche Qualifikationen sind für folgende Arbeiten notwendig?}

\begin{enumerate}
\item
  \textbf{VW E-Golf kommt zum Räderwechsel}

  \begin{itemize}
  \item
    Fachkundige unterwiesene Person
  \end{itemize}
\item
  \textbf{Lehrling soll VW E-Golf in eine andere Filiale fahren}

  \begin{itemize}
  \item
    Sensibilisierte Person
  \end{itemize}
\item
  \textbf{Beim Smart ED soll der Klimakompressor getauscht werden}

  \begin{itemize}
  \item
    Fall 1: Wenn nicht freigeschaltet ist $\to$ Stufe 2S
  \item
    Fall 2: Wenn freigeschaltet ist $\to$ Stufe 1S + Klimaschein
  \end{itemize}
\item
  \textbf{Tesla Model S braucht eine neue HV-Batterie}

  \begin{itemize}
  \item
    Fachkundige Person
  \end{itemize}
\item
  \textbf{Nach Unfall lässt sich Mercedes Vito E-Cell nicht mehr über
  Service-Disconnect-Stecker spannungsfrei schalten}

  \begin{itemize}
  \item
    Fachkundige Person -- für Arbeiten unter Spannung stehenden
    HV-System
  \end{itemize}
\end{enumerate}

\section{Erkläre das Freischalten des
HV-Systems?}\label{erklaere-das-freischalten-des-hv-systems}

Kunde kommt mit E-Auto in die Werkstatt. Was passiert jetzt?

Beginn Gefahrübergang: mit Schlüsselübergabe + Werkstattauftrag
unterschrieben

\begin{enumerate}
\item
  Hütchen mit Magnet aufs Dach $\to$ Achtung Hochspannung!
\item
  Fahrzeug in Werkstatt fahren
\item
  Abschranken

  \begin{itemize}
  \item
    im Abstand $1~m$ um das Fahrzeug herum
  \item
    Gefahren absperren: Bedeutung von gelb/schwarzes Band - langfristig
    oder rot/weiß - kurzfristig
  \end{itemize}
\item
  rotes Schild auf die Windschutzscheibe kleben $\to$ Achtung
  Hochvolt! Nicht freigeschaltet, mit Namen und Telefonnummer
\item
  Zündung ausschalten und Zündschlüssel entfernen
\item
  Sicherheitshandschuhe prüfen und anziehen, Schutzbrille, lange
  Kleidung ohne Reißverschluss (Vgl. Körperwiderstand)
\item
  Minuspol der 12 V-Batterie abklemmen
\item
  Warten 5 Minuten $\to$ bis Kondensatoren sich entladen von Bordnetz
  und HV-System

  \begin{itemize}
  \item
    Sicherheitslinie wird geöffnet
  \item
    Trennrelais öffnen und schalten Spannung ab
  \item
    Kondensatoren entladen sich
  \end{itemize}
\item
  Service-Disconnect-Stecker der HV-Batterie abziehen, falls nicht
  vorhanden, Sicherheitslinie unterbrechen
\item
  Zündschlüssel und Disconnect-Stecker wegschließen (mind. 10 m
  Entfernung) und sichern gegen wieder einschalten
\item
  Messen mit dem Duspol

  \begin{itemize}
  \item
    Messgerät überprüfen: Hochspannungsbereich 230 V-Steckdose,
    Niederspannungsbereich 12 V-Batterie
  \item
    Messung durchführen: am Inverter alle drei Phasen messen
  \item
    Messgerät überprüfen (gleiche Spannungsquellen):
    Hochspannungsbereich 230 V, Niederspannungsbereich 12 V-Batterie
  \item
    Spannungsfreiheit festgestellt
  \end{itemize}
\item
  Dokumentieren über das Freischalten (Was wurde gemacht? Wie wurde das
  Fahrzeug übergeben?)
\item
  weißes Schild auf die Windschutzscheibe kleben $\to$ Fahrzeug ist
  spannungsfrei, Fahrzeug gegen Wiedereinschalten gesichert, mit Namen
  und Telefonnummer
\item
  Jetzt kann gearbeitet werden.
\end{enumerate}

Pilotlinie, Interlock, Sicherheitslinie (Niedervolt)

\textbf{Wie hoch ist der Körperwiderstand ab einer Spannung von 100 V?}

Berührung der beiden Batteriepole

\begin{itemize}
\item
  $1000~\Omega$ von Hand-zu-Hand oder Hand-zu-Fuß
\item
  $450~\Omega$ von Hand-zu-Brust
\item
  $\text{Körperstrom} = \frac{400~V}{1000~\Omega} = 400~mA$ der
  absolut tödlich wäre.
\end{itemize}

Ein Strom von beispielsweise $50~mA$ kann zur Muskelverkrampfung
führen, sodass sich die Person nicht mehr selbstständig von der
Spannungsquelle lösen kann. Schon nach wenigen Sekunden könnte es auch
bei niedrigen Strömen zu tödlichen Unfällen kommen. (Einwirkzeit)

\textbf{Sicherheitshandschuhe prüfen} (rot, bis $1000~V$, vor jedem
Freischalten)

\begin{itemize}
\item
  Handschuh aufdrehen und auf Dichtheit prüfen oder Prüfgerät
\item
  keine Verfärbung oder Punkte
\item
  Aufdruck vollständig
\end{itemize}
