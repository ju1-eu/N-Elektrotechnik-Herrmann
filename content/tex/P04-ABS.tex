%ju 16-Dez-22 P04-ABS.tex
\textbf{1. Nennen Sie mindestens 4x Aspekte, die Sie mit dem Kunden zur
Eingrenzung des Fehlers besprechen sollten.}

\begin{figure}[!ht]% hier: !ht
\centering
\includegraphics[width=0.6\textwidth]{images/P06-ABS/P06-ABS-6.pdf}
\caption{Aufgabe 1}
%\label{fig:}%% anpassen
\end{figure}

\textbf{Leuchtende Kontrollleuchte: ESP an und ABS an}

\begin{enumerate}
\item
  Wurde etwas am Fahrzeug gemacht
\item
  Wann ist der Fehler aufgetreten? Starten, Fahrt
\item
  Tritt der Fehler sporadisch auf oder permanent
\item
  Gibt es Beeinträchtigung beim Fahren
\item
  Gibt es bauliche Veränderungen
\item
  Gab es einen Unfall
\item
  Seit wann ist der Fehler aufgetreten
\end{enumerate}

\textbf{2. Erklären Sie dem Kunden die Unterschiede von ABS, ASR, ESP}

\begin{itemize}
\item
  \textbf{ABS} verhindert bei einer Gefahrenbremsung das Blockieren der
  Räder, um eine Lenkbarkeit des Fahrzeugs beizubehalten.
\item
  \textbf{ASR} verhindert beim Anfahren den Schlupf an den
  Antriebsrädern durch Leistungsdrosselung oder Eingriff der
  Bremsanlage.
\item
  \textbf{ESP} verhindert ein Ausbrechen des Fahrzeugs um die Hochachse
  durch gezielten Bremseingriff.
\end{itemize}

\textbf{3. Erkläre Individualregelung, nenne Vorteile und Nachteile}

\textbf{Erkläre die Select-Low-Regelung, nenne Vorteile und Nachteile}

\textbf{Individualregelung,} sie kann den Bremsdruck für jedes einzelne
Rad regeln.

\begin{itemize}
\item
  Vorteil: sie kann den größtmöglichen Bremsdruck für jedes einzelne Rad
  anlegen
\item
  Nachteil: Giermoment steigt an, d.h. das Drehen um die Hochachse
\end{itemize}

\textbf{Select-Low-Regelung} regelt den Bremsdruck Achsweise

\begin{itemize}
\item
  Vorteil: weniger Gierrate, tendenziell bricht das Fahrzeug nicht so
  schnell aus
\item
  Nachteil: etwas längerer Bremsweg
\end{itemize}

\textbf{4. Da der Kunde ein Fahrzeug benötigt, möchte er sein Fahrzeug
nicht in der Werkstatt belassen.}

\begin{figure}[!ht]% hier: !ht
\centering
\includegraphics[width=0.6\textwidth]{images/P06-ABS/P06-ABS-5.pdf}
\caption{Aufgabe 4}
%\label{fig:}%% anpassen
\end{figure}

\textbf{Beurteilen Sie den Sachverhalt und welche Empfehlungen geben Sie
dem Kunden.}

\textbf{Beurteilung} Fahrzeug muss stehen bleiben, weil es nicht
verkehrssicher ist.

\textbf{Empfehlungen}

\begin{itemize}
\item
  Hol- und Bringservice
\item
  E-Bike
\item
  Fahrzeugersatz Mietwagen, um Geld zu verdienen (Achtung: Leihwagen ist
  kostenlos)
\end{itemize}

\newpage

\textbf{5. 5x Fehler sind im Fehlerspeicher hinterlegt. Welche möglichen
Ursachen können für jeden einzelnen Fehler vorliegen?}

\textbf{Fehler im Fehlerspeicher}

\begin{enumerate}
\item
  \textbf{Steuergerät ESP - Versorgungsspannung zu niedrig}
\item
  \textbf{Bremslichtschalter - Kurzschluss oder Defekt}
\item
  \textbf{Raddrehzahlsensor vorne links - elektrische Störung}
\item
  \textbf{Drehzahlsensoren Vorderachse - Wert unplausibel}
\item
  \textbf{Systemfehler - Folgefehler}
\end{enumerate}

\textbf{Ursachen}

\begin{enumerate}
\item
  Batterie
\item
  Defekte Bremslichtschalter oder Leitungsunterbrechung (Kurzschluss)
\item
  Raddrehzahlsensor vorne links Defekt, oder Multipolrad defekt
\item
  Defekter Sensor oder defektes Impulsrad, Übergangswiderstand, falsche
  Reifen (Abrollumfang)
\item
  Ursache sind die Fehler vorher
\end{enumerate}

\textbf{6. Formuliere den Text für den Werkstattauftrag}

Kundenbeanstandung: ABS- und ESP-Lampe leuchtet

Prüfarbeiten: Kundenbeanstandung prüfen, Fehler diagnostizieren

(Bemerkung: Werkvertrag erfüllt und Kunde muss zahlen)

\newpage

\textbf{7. Nenne Testwerkzeuge und Unterlagen für die Diagnosearbeiten}

\begin{enumerate}
\item
  Diagnosetester
\item
  Oszilloskop
\item
  Stromlaufplan
\item
  Adapter-Kabel für Messungen
\item
  Batterie Ladegerät
\item
  Persönliche Schutzausrüstung nach UVV
\end{enumerate}

\textbf{8. Zusätzlich zum Auslesen der Fehlercodes ermöglichen viele
Diagnosetester noch, sogenannte Istwerte anzuzeigen.}

\textbf{Ist-Werte aus dem ABS/ESP Steuergerät des Kundenfahrzeugs.}

\begin{enumerate}
\item
  \textbf{Steuergeräte Spannungsversorgung: 5,16 V}
\item
  \textbf{Spannung Ventilrelais: 5,19 V}
\end{enumerate}

\textbf{Bewerten Sie die beiden Istwerte und begründen Sie Ihre
Entscheidung.}

\textbf{Bewerten:}

\begin{enumerate}
\item
  Steuergeräte Spannungsversorgung: nicht i.O.
\item
  Spannung Ventilrelais: nicht i.O.
\end{enumerate}

\textbf{Begründen:} Versorgungsspannung kommt über Klemme 30 oder 15,
die Batteriespannung (Quellspannung) sollte bei 12 V liegen. Beim
Ventilrelais erwarten wir Quellspannung der Batterie.

\newpage

\textbf{9. Zur Bearbeitung des Fehlercodes Steuergerät ESP
Spannungsversorgung zu niedrig ist eine Überprüfung der
Spannungsversorgung des ESP Steuergerätes notwendig.}

\textbf{Beschreiben Sie die vollständige Überprüfung der
Spannungsversorgung (plus- und masseseitig) des ABS/ESP Steuergerätes
ohne Prüfadapter am Steuergerätestecker.}

\textbf{Benennen Sie die ausgewählten Messpunkte und Sicherung.
Markieren Sie diese Messpunkte und Sicherungen zusätzlich im
Schaltplan.}

\textbf{Begründen Sie Ihre Auswahl der Messpunkte.}

\begin{enumerate}
\item
  Ja, Quellspannung messen (auf Polklemmen, besser Bat. UB unter Last)
\item
  Verbraucher Uk (Klemmenspannung) ca. UB
\item
  Stecker vom Steuergerät (+) zum Massepunkt ca. UB
\item
  Uv (Spannungsfall) minusseitig \textless{} 0,5V, Stecker vom
  Steuergerät (-) nach Masse
\item
  Uv (Spannungsfall) plusseitig \textless{} 0,5V, Stecker vom
  Steuergerät (+) nach Bat. +
\end{enumerate}

\newpage

\textbf{10. Die nachfolgenden Abbildungen zeigen zwei Spannungsmessungen
im Bereich der Spannungsversorgung des ABS/ESP Steuergerätes mit einem
Prüfadapter.}

\begin{figure}[!ht]% hier: !ht
\centering
\includegraphics[width=0.4\textwidth]{images/P06-ABS/P06-ABS-4.pdf}
\caption{Aufgabe 10 Messung A}
%\label{fig:}%% anpassen
\end{figure}

\begin{figure}[!ht]% hier: !ht
\centering
\includegraphics[width=0.4\textwidth]{images/P06-ABS/P06-ABS-3.pdf}
\caption{Aufgabe 10 Messung B}
%\label{fig:}%% anpassen
\end{figure}

\textbf{Welche Messungen werden hier dargestellt und wie lauten ihre
Befunde zu diesen Messungen?}

\textbf{Begründen Sie Ihren Befund.}

\textbf{Messung A}

\begin{itemize}
\item
  Versorgungsspannung Steuergerät messen
\item
  Erwartung: 12 V Quellspannung
\end{itemize}

\textbf{Messung B}

\begin{itemize}
\item
  Übergangswiderstand 7,10 V
\item
  Messung: Spannungsverlust plusseitig zwischen Sicherung und
  Steuergerät
\item
  Potenzialdifferenz: 0 V = kein Spannungsverlust wäre i.O.
\end{itemize}

\textbf{11. Zur Erfassung der Raddrehzahl werden passive oder aktive
Drehzahlsensoren eingesetzt.}

\textbf{Beschreiben Sie den Aufbau und das Funktionsprinzip dieser
beiden Sensortypen zur Raddrehzahlerfassung.}

\textbf{Passiver Drehzahlsensor:} Induktivsensor

\begin{itemize}
\item
  \textbf{Aufbau:} Sensor, Spule, Impulsring
\item
  \textbf{Funktion:} Wir erzeugen das Magnetfeld am Sensor und schneiden
  es durch den Impulsring (Sinussignal), keine eigene
  Spannungsversorgung
\end{itemize}

\textbf{Aktiver Drehzahlsensor:} Hallsensor

\begin{itemize}
\item
  \textbf{Aufbau:} Messzelle - Hallsensor, Multipolring
\item
  \textbf{Funktion:} eigene Spannungsversorgung, Rechteckssignal,
  Drehrichtungserkennung: 2x Hallzellen
\end{itemize}

\newpage

\textbf{12. Im Fehlerspeicher des Kundenfahrzeugs ist der Fehlereintrag
Drehzahlsensoren Vorderachse Wert unplausibel hinterlegt.}

\textbf{Welche Bauart von Raddrehzahlsensoren ist laut Schaltplan in
Kundenfahrzeug verbaut?}

\textbf{Begründen Sie Ihre Antwort}

\textbf{Erläutern Sie die Vorteile der im Fahrzeug verbauten
Raddrehzahlsensoren.}

Passiver Drehzahlsensor, weil keine eigene Spannungsversorgung.

\textbf{Vorteile}

\begin{enumerate}
\item
  Günstiger in der Herstellung
\item
  Nicht so Fehleranfällig
\item
  Braucht keine Spannungsversorgung
\item
  einfacher zu prüfen
\end{enumerate}

\textbf{13. Benennen und beschreiben Sie alle möglichen Prüfschritte zur
Eingrenzung des Fehlercodes (Drehzahlsensoren Vorderachse, Wert
unplausibel.)}

\textbf{Berücksichtigen Sie dabei alle für den Fehlercode möglichen bzw.
verantwortlichen Komponenten und benennen Sie notwendige Messpunkte.}

\textbf{Prüfschritte}

\begin{enumerate}
\item
  Sichtprüfung
\item
  Oszilloskop Signal messen
\item
  Sensorbild
\item
  Luftspalt
\item
  Impulsring
\end{enumerate}

\textbf{Verantwortliche Komponenten:}

\begin{enumerate}
\item
  Radlager
\item
  Luftspalt zu groß
\end{enumerate}

\newpage

\textbf{14. Das nachfolgende Signalbild zeigt ein sogenanntes
Manchester-Protokoll (Rechtecksignal)}

\textbf{Erläutern Sie, weshalb bei der Signalaufnahme des
Raddrehzahlsensors die Kopplung des Oszilloskops in AC erfolgt und die
Trigger Flanke negativ eingestellt wird.}

\begin{figure}[!ht]% hier: !ht
\centering
\includegraphics[width=0.6\textwidth]{images/P06-ABS/P06-ABS-2.pdf}
\caption{Aufgabe 14}
%\label{fig:}%% anpassen
\end{figure}

Rechtecksignal

\begin{itemize}
\item
  Beim Hallsignal kann ein positives oder negatives Signal sein.
\item
  Wenn ich nur Positiv auswähle, dann kann ich nicht das Signal
  beurteilen.
\end{itemize}

\newpage

\textbf{15. Häufig verfügen moderne Fahrzeuge über Raddrehzahlsensoren,
die eine Drehrichtungserkennung besitzen.}

(sog. Hallsensoren mit erweiterten Funktionen bzw. Multipolring)

\textbf{Beschreiben Sie das Funktionsprinzip durch das Steuergerät.}
(Vorwärts-Rückwärts-Erkennung bzw. Stillstands-Erkennung)

\textbf{Benennen Sie mind. zwei weitere Systeme, die eine
Drehrichtungserkennung der Raddrehzahlsensoren verarbeiten bzw.
benötigen.}

\begin{figure}[!ht]% hier: !ht
\centering
\includegraphics[width=0.4\textwidth]{images/P06-ABS/P06-ABS-7.pdf}
\caption{Aufgabe 15}
%\label{fig:}%% anpassen
\end{figure}

\begin{figure}[!ht]% hier: !ht
\centering
\includegraphics[width=0.4\textwidth]{images/P06-ABS/P06-ABS-8.pdf}
\caption{Aufgabe 15}
%\label{fig:}%% anpassen
\end{figure}

\textbf{Hallsensor Drehrichtungserkennung,} je nachdem welche Hallzelle
als Erstes die Poländerung erreicht, kann erkannt werden welche Richtung
vorliegt.

\textbf{Stillstands-Erkennung,} bei keiner Poländerung steht das
Fahrzeug.

\textbf{Weitere Systeme, die eine Drehrichtungserkennung benötigen}

\begin{enumerate}
\item
  Parksensoren bei rückwärts rollen am Berg
\item
  Elektrische Differenzialsperre
\end{enumerate}

\newpage

\textbf{16. Der Datenaustausch zwischen den Steuergeräten und dem Sensor
erfolgt über einen CAN-Datenbus}

\textbf{Nennen Sie mind. 6 Konstruktionsmerkmale des im Fahrzeug
verwendeten CAN-Datenbus.}

\textbf{Class C: CAN-Antriebsbus - ABS}

\begin{enumerate}
\item
  nicht Eindrahtfähig, weil sicherheitsrelevant
\item
  Selbstabschirmung durch verdrillte Kabel
\item
  Potenzialverteiler
\item
  Abschlusswiderstände im Potenzialverteiler
\item
  CAN-High und CAN-Low
\item
  Geschwindigkeit: 1 Mbit/s
\end{enumerate}

\newpage

\textbf{17. Die nachfolgende Abbildung zeigt den Signalverlauf des
CAN-Datenbus-Systems}

\begin{enumerate}
\item
  \textbf{Kennzeichen sie in der Abbildung folgende Werte}

  \begin{itemize}
  \item
    \textbf{CAN-High und CAN-Low Signal}
  \item
    \textbf{In beiden Datenleitungen sowohl den dominanten und
    rezessiven Zustand}
  \end{itemize}
\item
  \textbf{Ergänzen Sie die fehlenden Spannungsangaben (1 / 2 / 3) in der
  Abbildung.}
\end{enumerate}

\begin{figure}[!ht]% hier: !ht
\centering
\includegraphics[width=0.6\textwidth]{images/P06-ABS/P06-ABS-1.pdf}
\caption{Aufgabe 17}
%\label{fig:}%% anpassen
\end{figure}

\textbf{Spannungsangaben}

\begin{enumerate}
\item
  4 V
\item
  2,5 V
\item
  1 V
\end{enumerate}

\begin{figure}[!ht]% hier: !ht
\centering
\includegraphics[width=0.6\textwidth]{images/P06-ABS/P06-ABS-1-Loesung.pdf}
\caption{Aufgabe 17 Lösung}
%\label{fig:}%% anpassen
\end{figure}
